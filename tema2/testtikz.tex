% !TeX program = xelatex
% !TeX encoding = UTF-8
\documentclass{article} 
    \usepackage{fontspec}
    
    
    \usepackage{amsmath,amssymb,amsthm}
    \setlength{\parindent}{0pt}
    
    \usepackage{polyglossia}   %% загружает пакет многоязыковой вёрстки
    \setdefaultlanguage[spelling=modern]{russian}  %% устанавливает главный язык документа
    \setotherlanguage{english} %% объявляет второй язык документа
    \defaultfontfeatures{Ligatures={TeX},Renderer=Basic}  %% свойства шрифтов по умолчанию
    \setmainfont[Ligatures={TeX,Historic}]{Times New Roman} %% задаёт основной шрифт документа
    
    
    \setsansfont{Arial}
    \setmonofont{Courier New}
    
    \newfontfamily{\cyrillicfont}{Times New Roman}
    \newfontfamily{\cyrillicfontrm}{Times New Roman}
    \newfontfamily{\cyrillicfontsf}{Arial}
    \newfontfamily{\cyrillicfonttt}{Courier New}


\usepackage{xcolor}

\newtheorem{theorem}{Теорема}

\usepackage{tikz}
\usepackage{tkz-graph}

\usetikzlibrary{shapes,calc,positioning,graphs, arrows,patterns,through,fadings,mindmap}

\listfiles 


\begin{document}
    \begin{theorem}
        Всё хорошее когда-нибудь кончается.
        
     \end{theorem}   
    
      
  $ \foreach \i / \k in {1/a, 2/b, 3/c} {( \i: \k) } $
   
%   \foreach \i / \k in {1/a, 2/b, 3/c} {( \i: \k) }
   
    \foreach \i  in {1/a, 2/b, 3/c} {( \i: afa) }
    
    
    
\foreach \i in {1,2, ..., 4} {\i --} \\
 \foreach \i in {4 , ..., 1} {\i \ $ \rightarrow $ } \\
\foreach \i in {a, ..., d} {\i , } \\
\foreach \i in {1, 1.5 , ..., 5.8} {-\i -} 

% проверьте пакет amsmath
 $ \foreach \i / \j in {1/a, 2/b, 3/c}  {\dfrac{\i}{\j} }$    
    
    
 \foreach \i [ evaluate = \i as \result ] %
in {1 + 2, 10 * 3, 2^3} %
{$ \i = \result \quad $} %
    
\begin{figure}
\centering
\begin{tikzpicture}[scale=3]
\draw [thick, green!40!black ] %
(0, 0) -- +(0:1 cm) -- +(60:1 cm) -- cycle ;
\end {tikzpicture}




\caption {a picture in a figure }
\end {figure}



\tikz %
\draw %
(0, 0) circle [ radius = 5mm ];
\\[1 em]

\tikz %
\draw %
(0, 0) circle %
[ %
x radius = 5mm , %
y radius = 3mm %
];


\begin{figure}
\centering
	\begin{tikzpicture}
    \tikzstyle{вопрос} = [ellipse] ;	
    \tikzstyle{совет} = [rounded rectangle,fill=gray!10] ;	
    \graph[nodes={align=center, rectangle,draw=blue}, grow down sep, branch left sep] {
        Железяка работает?[вопрос] ->
        { 
            "Вроде, да"-> Не трогай! [совет],
            Нет -> Ты её трогал? [вопрос] -> 
            {
                "Нет, не трогал" -> "Тогда\dots{} не знаю\dots{}" [совет] 
                ,
                "Трогал" -> "Верни все, как было!" [совет]
                } 
                } -> 
              Успокойся! [совет]                 };
  \end{tikzpicture}
\caption{Визуализация графов}
\end{figure}


\noindent\tikz %
\draw %
{ [dashed] (0, 0) -- (1, 0) }
{ [ rounded corners,very thick,draw=green ] -- (1, 1) } %
-- (0, 1) -- cycle ; \\[1 em]
\tikz %
\draw %
(0, 0) -- (1, 0) %
[ color = red ] -- ++(90:1);

\begin{figure}[th]
    \centering
    \begin{tikzpicture}[scale=0.2]
        \draw (5,5) -- ++(1,0) 
         -- ++ (0,1) -- ++(1,0) rectangle +(-1,0.2) +(0,0) 
         -- ++ (0,1) -- ++(1,0) rectangle +(-1,0.2) +(0,0) 
         -- ++ (0,1) -- ++(1,0) rectangle +(-1,0.2) +(0,0) 
         -- ++ (0,1) -- ++(1,0) rectangle +(-1,0.2) +(0,0) 
         ; % завершает путь 
          
     \end{tikzpicture}
     \caption{лестница без циклов}
\end{figure}
    \begin{figure}[th]
        \centering
        \begin{tikzpicture}[scale=0.2]
        \draw  (5,5) -- ++(1,0) 
        \foreach \ii in {1,2, ...,10}
          { -- ++ (0,1) -- ++(1,0) rectangle +(-1,0.2) 
            +(0,0) 
          } ;
      
        
        \end{tikzpicture}
        \caption{Лестница c циклами и прямоугольниками}
    \end{figure}
    
\begin{figure}[th]
    \centering
    \begin{tikzpicture}[scale=0.2]
    \draw (5,5) -- ++(1,0) 
    -- ++ (0,1) -- ++(1,0) {[rounded corners=1] +(0,0.2) -| +(-1,0)} +(0,0) 
    -- ++ (0,1) -- ++(1,0) {[rounded corners=2] +(0,0.2) -| +(-1,0)} +(0,0) 
    
    -- ++ (0,1) -- ++(1,0) rectangle +(-1,0.2) +(0,0) 
    -- ++ (0,1) -- ++(1,0) rectangle +(-1,0.2) +(0,0) 
    -- ++ (0,1) -- ++(1,0) rectangle +(-1,0.2) +(0,0) 
    ; % завершает путь 
    
    \end{tikzpicture}
    \caption{лестница без циклов с прямоугольными зигзагами}
\end{figure}
    
\begin{figure}[th]
    \centering
    \begin{tikzpicture}[scale=1]
    \draw (5,5) -- ++(2,0) 
    -- ++ (0,1) -- ++(1,0) {[rounded corners=1] +(0,0.2) -| +(-1,0)} +(0,0) 
    -- ++ (0,1) -- ++(1,0) {[rounded corners=2] +(0,0.2) -| +(-1,0)} +(0,0) 
    
    -- ++ (0,1) -- ++(1,0) rectangle +(-1,0.2) +(-.5,0.5) circle [x radius=4mm, y radius=3mm]    +(0,0) 
     -- ++ (0,1) -- ++(1,0) rectangle +(-1,0.2) +(-1,0.5) circle [x radius=4mm, y radius=3mm, start angle=-160,end angle=30]    +(0,0) 
     
    -- ++ (0,1) -- ++(1,0) rectangle +(-1,0.2) +(0,0) 
    -- ++ (0,1) -- ++(1,0) rectangle +(-1,0.2) +(0,0) 
    ; % завершает путь 
    
    \end{tikzpicture}
    \caption{лестница без циклов с прямоугольными зигзагами}
\end{figure}    

\begin{figure}[th]
    \centering
    \begin{tikzpicture}[scale=1,font=\tiny]
    \path [draw,left color=red,right color=blue,line width=0.7pt] (5,5) -- ++(2,0) node [anchor=south west]{низ}
    -- ++ (0,1) -- ++(1,0) {[rounded corners=1] +(0,0.2) -| +(-1,0)} +(0,0) 
    -- ++ (0,1) -- ++(1,0) {[rounded corners=2] +(0,0.2) -| +(-1,0)} +(0,0) 
    
    -- ++ (0,1) --   ++(1,0) rectangle  +(-1,0.2) +(-.5,0.5) circle [x radius=4mm, y radius=3mm]    +(0,0) 
    -- ++ (0,1) -- ++(1,0) grid [ystep=0.6mm] +(-1,0.2) +(-1,0.5) circle [x radius=4mm, y radius=3mm, start angle=-160,end angle=30]    +(0,0) 
    
    -- ++ (0,1) -- ++(1,0) grid [xstep=3mm,ystep=0.4mm] +(-1,0.2) +(0,0) 
    -- ++ (0,1) -- ++(1,0) rectangle node[above]{верх} +(-1,0.2) +(0,0) 
    ; % завершает путь 
    
    \end{tikzpicture}
    \caption{лестница без циклов с прямоугольными зигзагами}
\end{figure}    



\begin{tikzpicture}
    \draw (1,0) -- (0,0) -- (0,1) ; \draw (0,1) to [bend left=20] (1,0) node[above,sloped]{надпись}  ;
    
\end{tikzpicture}


    
\end{document}
