% !TeX spellcheck = ru_RU
% !TeX program = xelatex
% !TeX encoding = UTF-8 Unicode

\documentclass[xcolor=dvipsnames]{beamer}
\usepackage{xltxtra}
\usepackage{fontspec}
\usepackage{polyglossia}
% \usepackage{minted}
\setmainlanguage[babelshorthands=true]{russian}
\usepackage{xcolor}

\usepackage{calc}
\usepackage{listings}
\usepackage{ulem}


% babel shorthands are:
%  "--- Cyrillic emdash in plain text.
%  "--~ Cyrillic emdash in compound names (surnames).
%  "--* Cyrillic emdash for denoting direct speech.
% "~ unbreakable hype. Example: как в 90"~х годах

\setotherlanguage{english}



\setmainfont{Cambria}
%\setromanfont{Times New Roman}
\setsansfont{Arial}
\setmonofont{Courier New}



% \newfontfamily{\cyrillicfontrm}{Times New Roman}
\newfontfamily{\cyrillicfonttt}{Courier New} 



\usetheme{Warsaw}
\usepackage{graphicx}

\usepackage{wrapfig}
% !TeX root = rand_text.tex

\usepackage{xcolor}

\usepackage{tikz}
\usetikzlibrary{shapes,calc,positioning,graphs, arrows,patterns,through}


%\definecolor{gray50}{gray}{0.5}
\definecolor{myblue}{RGB}{12,131,200}
\definecolor{myred}{RGB}{200,100,100}
\definecolor{mydeepblue}{HTML}{1F2FAA}
    
%\tikzset{everypicture/.style=thick, rounded corners=5}
    

% !TeX root = rand_beamer.tex


\author{Илья Кочергин}
\institute{{\small Сетевая Академия}\\ \includegraphics[width=1.5cm,clip,trim=2mm 10mm 2mm 2mm]{vert_logo_lanit.pdf}}

\title{Новшества языка Java 8}
\begin{document}
    \begin{frame}
        \titlepage
    \end{frame}
    \definecolor{mygreen}{rgb}{0,0.6,0}
    \definecolor{mygray}{rgb}{0.5,0.5,0.5}
    \definecolor{mymauve}{rgb}{0.58,0,0.82}
    \lstset{ %
        backgroundcolor=\color{white},   % choose the background color; you must add \usepackage{color} or \usepackage{xcolor}
        basicstyle=\footnotesize,        % the size of the fonts that are used for the code
        breakatwhitespace=false,         % sets if automatic breaks should only happen at whitespace
        breaklines=true,                 % sets automatic line breaking
        captionpos=b,                    % sets the caption-position to bottom
        commentstyle=\color{mygreen},    % comment style
        deletekeywords={...},            % if you want to delete keywords from the given language
        escapeinside={\%*}{*)},          % if you want to add LaTeX within your code
        extendedchars=true,              % lets you use non-ASCII characters; for 8-bits encodings only, does not work with UTF-8
        frame=single,                    % adds a frame around the code
        keepspaces=true,                 % keeps spaces in text, useful for keeping indentation of code (possibly needs columns=flexible)
        keywordstyle=\color{blue},       % keyword style
        language=Java,                 % the language of the code
        otherkeywords={*,...},            % if you want to add more keywords to the set
        numbers=left,                    % where to put the line-numbers; possible values are (none, left, right)
        numbersep=5pt,                   % how far the line-numbers are from the code
        numberstyle=\tiny\color{mygray}, % the style that is used for the line-numbers
        rulecolor=\color{black},         % if not set, the frame-color may be changed on line-breaks within not-black text (e.g. comments (green here))
        showspaces=false,                % show spaces everywhere adding particular underscores; it overrides 'showstringspaces'
        showstringspaces=false,          % underline spaces within strings only
        showtabs=false,                  % show tabs within strings adding particular underscores
        stepnumber=2,                    % the step between two line-numbers. If it's 1, each line will be numbered
        stringstyle=\color{mymauve},     % string literal style
        tabsize=2,                       % sets default tabsize to 2 spaces
        title=\lstname                   % show the filename of files included with \lstinputlisting; also try caption instead of title
    }
\begin{frame}[fragile]
\frametitle{Блок try-with-resources [Java 7]}
%\begin{block}{Ресурсы}
    \begin{itemize}
%        \small
    \item Ресурсы "--- объекты классов, реализующие интерфейс java.lang.AutoСloseable
    
    \item Они представляют в программе открытые файлы, сетевые соединения, для которых нужно не забывать вызвать метод close(), в противном случае возможно постепенное исчерпание ресурсов ОС
    
    \item \uline{Ресурсы}, объявленные в \alert{круглых скобках} после ключевого слова try, \uline{автоматически закрываются} при завершении работы блока  try любым образом (штатно или по исключению)
\end{itemize}

%\end{block}
\begin{lstlisting}
try%*\alert{(}*)
  %*\uline{Scanner in}*) =
     new Scanner(Paths.get("/usr/share/dict/words"));
  %*\uline{PrintWriter out}*) = new PrintWriter("/tmp/out.txt")
%*\alert{)}*)  {
    while (in.hasNext())
        out.println(in.next().toLowerCase());
}catch(IOExeption e) { %*\ldots*) }
\end{lstlisting}
\end{frame}
\end{document}

